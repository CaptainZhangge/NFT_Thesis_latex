\chapter{Abstract}



New services and applications are causing an exponential increase in internet traffic.  In a few years, current fiber optic communication system infrastructure will not be able to meet this demand because fiber nonlinearity dramatically limits the information transmission rate. Eigenvalue communication is considered an emerging paradigm in fiber optics communications that could potentially overcome these limitations. It relies on a mathematical technique called ``inverse scattering transform'' or ``\ac{NFT}'' to exploit the ``hidden'' linearity of the \acl{NLSE} as the master model for signal propagation in an optical fiber.
One of the rapidly evolving \ac{NFT}-based communication techniques is called \ac{NFDM}.
% This modulation method encodes the data on the  nonlinear spectrum associated by the NFT to a time domain signal.
Being still in its infancy, \ac{NFDM} systems still have some practical limitations. One of these limitations is the lack of polarization division multiplexing.

The results presented in this thesis address this problem.
% First the theory of single polarization NFT is reviewed, and a generalized channel model that accounts for fiber loss and noise is introduced. The applicability of the NFT over this channel is evaluated through numerical simulations, demonstrating that even using this approximate model the nonlinear spectrum is less distorted by nonlinearities when compared to the linear Fourier spectrum.
First the structure of an \ac{NFDM} system using the discrete nonlinear spectrum is described. The particular design aspects of this system are then discussed in details.

Subsequently, the theoretical tools describing the NFT for the Manakov system describing the evolution of a dual polarization signal in a \acl{SMF}  are presented. Using these tools the discrete \ac{NFDM} system is extended to the the dual polarization case.
Finally, the results of the first experimental transmission of a dual polarization \ac{NFDM} system are presented. A transmission of up to 373.5 km with bit error rate less than the hard-decision forward error correction threshold has been achieved.

The results presented demonstrate that dual-polarization \ac{NFT} can work in practice and enable an increased spectral efficiency in \ac{NFT}-based communication systems, which are currently based on single polarization channels.

