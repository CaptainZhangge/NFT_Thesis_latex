\section{NFT in the presence of losses and noise}
The channel model described by the NLSE in \eqref{eq:NLSE} is a simplistic
model and does not model properly existing fiber optics channel. Indeed, real
channels are affected by noise and have losses.
A more accurate form of the NLSE is

%% .EQ. NLSE WITH LOSS %%
\begin{equation}\label{eq:NLSE_with_loss}
\frac{\partial \fld(\ttm,\ssp)}{\partial \ssp} =
-i \frac{\dispersion}{2} \frac{\partial^2 \fld(\ttm,\ssp)}{\partial \ttm^2}
+ g(\ssp)\fld(\ttm,\ssp)
+ i\nonlinfact |\fld(\ttm,\ssp)|^2 \fld(\ttm,\ssp)
+ \asenoise(\ttm,\ssp)
\end{equation}

where $\asenoise(\ttm,\ssp)$ is the noise term and
\begin{equation}
 g(\ssp) = - \dfrac{\loss}{2} + g'(\ssp)
\end{equation}
is the gain-loss profile given by a combination of the losses, characterized by
the attenuation coefficient of the fiber $\loss$ and the gain profile
$g'(\ssp)$, which depend on the amplification scheme used. We assume that the
fiber link is divided in spans of length $\spanl$ with the same amplification
scheme applied to each span.

By using advanced amplification techniques, such as DRA [REF], it is possible
to obtain an almost flat gain profile over the entire link thus making $g(\ssp)
= 1 $. In this way it is possible to make \eqref{eq:NLSE_with_loss} very close
to the ideal lossless NLSE, but perturbed by noise, which for such
amplification scheme can be particularly problematic given the high RIN transfer
[REF].
When EDFA amplification is used instead, we have $g(\ssp) = - \frac{\loss}{2} $
for $0 < \ssp < \spanl$, where $L$ is the length of the fiber between two
amplifiers.

The presence of the two new terms makes \eqref{eq:NLSE_with_loss} not
integrable, so that the NFT method does not provide exact analytical solutions
anymore. Nonetheless, it has been shown that the NFT can still be used to .
The way to remove the loss term, is to use the loss path-averaged method [REF]
as follows: we first perform the change of variable
\begin{equation}
  \tilde{\fld}(\ttm,\ssp) \leftarrow{} \fld(\ttm,\ssp)G(z)^{\frac{1}{2}}
\end{equation}
with the total loss up to the point $\ssp$
\begin{equation}
  G(\ssp) = exp\left(2\int_{0}^{\ssp} g(\ssp') d\ssp' \right)
\end{equation}
in order to obtain
\begin{equation}\label{eq:NLSE_with_loss_after_change_var}
\pdv{\tilde{\fld}(\ttm,\ssp)}{\ssp} =
-i \frac{\dispersion}{2} \pdv[2]{\tilde{\fld}(\ttm,\ssp)}{\ttm}
+ i\nonlinfact(\ssp) \abs{\tilde{\fld}(\ttm,\ssp)}^2 \tilde{\fld}(\ttm,\ssp)
+ \asenoise(\ttm,\ssp)
\end{equation}
with
\begin{equation}
 \nonlinfact(\ssp) = \nonlinfact G(\ssp)
\end{equation}
is a space dependent nonlinear coefficient.
Finally by averaging this coefficient over one span length as
\begin{equation}
  \bar{\nonlinfact} = \dfrac{1}{\spanl}\int_{0}^{\spanl} \nonlinfact(\ssp) d\ssp
\end{equation}
we can write the \ac{LPA}~\ac{NLSE} as
\begin{equation}\label{eq:NLSE_LPA}
  \pdv{\tilde{\fld} (\ttm,\ssp)}{\ssp} =
  -i \frac{\dispersion}{2} \pdv[2]{\tilde{\fld}(\ttm,\ssp)}{\ttm}
  + i\bar{\nonlinfact}\abs{\tilde{\fld}(\ttm,\ssp)}^2 \tilde{\fld}(\ttm,\ssp)
  + \asenoise(\ttm,\ssp)
\end{equation}


In the following section we numerically analyze in more details the effect of
some system parameters (launch power, fiber attenuation, and additive white
Gaussian (AWG) noise) on the \ac{NFT} continuous spectrum with the aim of
comparing how the same effects affect the linear spectrum (discrete Fourier
transform (DFT)). This is done by considering  a simple scenario of a single
isolated Gaussian pulse propagating in a standard single mode fiber (SMF) and
measuring the difference between its continuous nonlinear spectrum before and
after transmission in terms of normalized mean squared error (NMSE). We show
the stronger potential of the nonlinear spectrum for preserving itself along
the transmission link when compared tothe DFT regardless of the presence of
non-idealities such as fiber loss and noise.  The \ac{NFT} is thus a promising
technique in scenarios where the spectrum preservation is desirable, such as
multi channel systems using frequency-orthogonal channels.

\iffalse % Already defined in previous sections
\section{Nonlinear Fourier Transform}
Assuming to operate in a normalized regime as the one used in
\cite{YaousefiPartI}, to obtain the \ac{NFT}
continuous spectrum \cnft{} starting from the normalized time waveform \nfld,
we first need to solve the
eigenvalue problem $Lv = \lambda v$, where $L$ is the
isospectral operator associated to the nonlinear Schr\"odinger equation (NLSE).
The eigenvalue problem can be
simplified to the system of nonlinear equations

\begin{equation}\label{eigenvalue.problem}
\deriv{\vv}{t} = \matP v, \qquad v(t_1, \lambda)e^{-j \lambda t_1}
\end{equation}
and must be solved for a specific $\lambda$ on the interval $[t_1,t_2]$ over
which \nfld is defined. By using
the resulting vector $v(t_2)$ we can obtain the two \ac{NFT} coefficients
$a(\lambda) = v_1(t_2)e^{-j\lambda t_2}$
and $b(\lambda) = v_2(t_2)e^{j\lambda t_2}$ from which it's possible to compute
the continuous spectrum of
the \ac{NFT}
$\hat{q}(\lambda) = \frac{b(\lambda)}{a(\lambda)}, \lambda \in \mathbb{R}$
The continuous eigenvalues $\lambda$ are related to the linear spectrum
frequencies $f$ by the relation
$\lambda = \pi f$. Because of this in the next sections we'll compute \cnft{}
over the interval $\left [-\pi
\frac{Fc}{2}, \pi \frac{Fc}{2} \right ]$, in such a way to guarantee a more
fairly comparison with the
linear
spectrum.

The transfer function of the channel is $H(\lambda, z) =
e^{-4j\lambda^2 z}$ and describes the evolution of the \ac{NFT} continuous
spectrum propagating for a
distance of $z$ as following
$\hat{q}(t, z) = H(\lambda, z)  \hat{q}(t, 0)$
\fi

\subsection{Simulation setup}

\begin{figure}[htbp]
  \centering
  \includegraphics[width=\textwidth]{./img/setup.eps}
  \caption{Simulation setup implemented in {Matlab\texttrademark}~and {VPI
Transmission Maker\texttrademark}.
Inset: \ac{NFT} and DFT spectra at input and output of the fiber lossless
channel.}
  \label{fig:setup}
\end{figure}
%
The numerical simulation setup is depicted in \figurename~\ref{fig:setup} and
has been performed in {Matlab\texttrademark}~and {VPI Transmission
Maker\texttrademark}. The transmitter generates a single Gaussian pulse with 50
ps full width at half maximum (FWHM). The simulation time window is 8~ns and the
bandwidth 2.56~THz. This choice guarantees that the pulse is well isolated,
condition required to properly compute the \ac{NFT}. An ideal laser with a
carrier frequency of 192.4~THz is used. The input waveform $\fld[i]$ is
transmitted over 1000 km (10$\times$100~km) of dispersion uncompensated SMF. The
fiber is assumed to have a dispersion $D = 17~ps/nm{\cdot}km$ and a nonlinear
coefficient $\gamma = 1.27~W^{-1}{\cdot}km^{-1}$. The fiber attenuation
coefficient is either set to $\alpha = 0$ (lossless, ideal case) or $\alpha =
0.2~dB/km$ (typical value for SMF). For a lossy channel, perfect attenuation
compensation has been achieved by using Erbium-doped fiber amplifiers (EDFAs)
after each span. EDFAs with no noise and with a noise figure of 3~dB are
considered in two different scenarios. In the simulations where noise is
present, the results have been averaged over 30 realizations and the standard
error is shown with error bars. The propagation in the fiber is simulated using
the split step Fourier method (SSFM) with an adaptive step size allowing a
maximum phase rotation of 0.01~$deg$. At the fiber output, an ideal coherent
receiver is assumed to give access to the full signal field $\fld[o]$.

The \ac{NFT} continuous spectra $\cnft[i]$ and $\cnft[o]$  with $\lambda\in
C^+$, of $\fld[i]$ and $\fld[o]$ respectively, are computed using the
Ablowitz-Ladik method to solve the \ac{NFT} eigenvalue problem as described
in~\cite{yousefi2014information}. When the channel is lossy and EDFA amplification is
employed, the \ac{NFT} is computed using the loss path average (LPA) method
described above. To compare input and output spectra, the channel inverse
transfer function $H^{-1}(\lambda, z=1)$ is used to compensate for the channel
propagation, $\cnft[i, est] = e^{4j\lambda^2}\cnft[o]$. The two spectra
$\cnft[i]$ and $\cnft[i,est]$ are then compared using the NMSE as a metric:

\begin{equation}
 NMSE = \dfrac{ \bigintsss_{\lambda}
 |\cnft[i] - \cnft[i,est]|^2}{\bigintsss_{\lambda}|\cnft[i]|^2}
\end{equation}

In order to benchmark the results, linear spectra are also calculated using
standard DFT and the same NMSE metric is applied. In this case, electrical
dispersion compensation (EDC) is applied to $\fld[o]$ as shown in
\figurename~\ref{fig:setup}.

\subsection{Results}

To investigate the impact of noise on the \ac{NFT} spectrum, an ideal Gaussian
pulse has been loaded with AWG noise corresponding to different levels of OSNR
defined in a 0.1~nm bandwidth. The \ac{NFT} and DFT spectra have then been
compared to their noiseless counterparts
%at the input of a lossless channel
% JUST ADD CONFUSION
for three different power levels. The dependency of DFT NMSE
and \ac{NFT} NMSE on the OSNR is depicted in \figurename~\ref{fig:osnrSweep}.
%The NMSE is inversely proportional to the OSNR for both transformations.
% IS IT USEFUL TO SAY SO?
For low powers the DFT and \ac{NFT} spectra have similar shapes and it can be
seen that they are similarly affected by noise. As the power increases to -12
dBm, the \ac{NFT} is more affected by noise leading to NMSE values greater than
those of the DFT.
% WE HAVEN'T SAID THAT THE TWO SPECTRUM ARE EQUAL FOR LOW POWERS
%
\begin{figure}[htbp]
\centering
    \includegraphics[width=.8\textwidth]{./img/figure_2.eps}
    \caption{NMSE between ideal and noise-loaded input spectra for \ac{NFT}
    and DFT as a function of the OSNR.}
    \label{fig:osnrSweep}
    % \caption{NMSE of FFT and \ac{NFT} spectra as a function of input power after 1000 km of transmission.}
\end{figure}

In \figurename~\ref{fig:powersweep}~(a) the NMSE as a function of launch power
is shown. In the lossless case, the DFT NMSE increases exponentially with the
power due to the impact of self phase modulation (SPM). For power levels above
-18~dBm, the relative error is higher than 10\% making the output spectrum
diverge significantly from the input one. Note that SPM impairments are already
so strong at this power, because the large guard time intervals used to account
for dispersive effects significantly decrease the average power, however,
nonlinear effects depend on the pulse peak power which is 22 dB higher than the
average. Above -9~dBm the error flattens out to NMSE $ = 4$, i.e. the maximum
NMSE between two spectra carrying the same energy. On the other hand, the
\ac{NFT} NMSE remains practically constant and below a value of $2\times10^{-6}$
for powers up to -10~dBm,
%\note[id=SG]{Error floor \ac{NFT}: 2.52e-6, DFT: 5.87e-6 obtained for power < 40dBm. It's
%not a real error floor since for higher powers we have a lower error at some point}
while for powers above
%this
value it slightly increases, possibly due to the numerical precision required
for the computation, as above -10~dBm most of the energy is transferred from the
\ac{NFT} continuous spectrum to the discrete
one~(\figurename~\ref{fig:powersweep}~(b)). Nevertheless the NMSE stays below a
value of $10^{-4}$, proving the theoretically expected higher tolerance of the
\ac{NFT} to nonlinear impairments. When fiber loss is taken into account and
compensated for by noiseless amplification, the DFT matching improves compared
to the lossless case as the effective nonlinearity is lowered by the presence of
the losses. Such an improvement is not visible for the \ac{NFT}, which performs
worse with respect to the lossless case due to the approximate equation used to
compute the \ac{NFT} when the channel is lossy and therefore non-integrable. The
NMSE exponential increase with power is consistent with~\cite{le2015nonlinear}.
Nonetheless, the spectral mismatch between the \ac{NFT} spectra is still lower
than that of the linear spectrum even in this non-ideal case.

In the last scenario considered, the impact of power variations and noise are
combined: the pulse has been transmitted over a lossy channel and EDFAs with a
noise figure of 3~dB have been used to compensate for the power loss. In
\figurename~\ref{fig:powersweep}~(a) the dependency of the NMSE as a function of
the launch power, and the corresponding OSNR at the receiver
%~\note[id=FDR]{We can fix this figure adding the OSNR as top axis,
%I can help with that in case}
, is shown. In the linear regime, the spectral matching increases with the power
(NMSE decreases) for both DFT and \ac{NFT}. Once the power is increased beyond
the linear transmission regime, the DFT NMSE worsens as nonlinearity impacts the
spectral matching. The behavior of the \ac{NFT} is similar, but in this case the
tolerance to nonlinearity is higher with the optimum launch power increased by
3~dB %\note[id=FDR]{Is that correct? Check the minima} . For even higher power
also the \ac{NFT} NMSE starts worsening. This is believed to be caused by the
use of the LPA approximation
%\note[id=FDR]{Are we sure about this?Aston people may complain :> This is what
% we say above, citing them. There is a penalty that has an exponential
%dependence on the power (red solid %curve of Figure 1)}
. Note that the peak for -6~dBm launch power corresponds to the point where the
first discrete eigenvalue appears, i.e., the point where the \ac{NFT} continuous
spectrum energy is slightly different from the \ac{NFT} total energy
%which causes the \ac{NFT} continuous spectrum to be ``peaky''
(\figurename~\ref{fig:powersweep}~(d)). Overall, the
NFT provides an NMSE decrease of a factor 2.85 compared to the DFT at the
respective optimum launch powers, proving its potential for nonlinear
transmission.
% Even tough we cannot directly map the NMSE metric to the $Q^2$, this shows how
% the \ac{NFT} used even in nonideal scenarios has the potential to provide improved performance compared to
% modulation that rely on the linear spectrum.
%
\begin{figure}[htbp]
  \centering
  \includegraphics[width=.8\textwidth]{./img/figure_1.eps}
   \caption{NMSE between input and output (1000 km) spectrum for \ac{NFT}
   and DFT (a),(c) and percentage of energy
in the continuous \ac{NFT} spectrum (b),(d) as a function of the launch power
for noiseless (a),(b) and noisy
(c),(d) amplification.}
    % \caption{NMSE of FFT and \ac{NFT} spectra as a function of input power
    % after 1000 km of transmission.}
    \label{fig:powersweep}
\end{figure}

\begin{figure}[htbp]
  \centering
   \includegraphics[width=.8\textwidth]{./img/figure_3.eps}
   \caption{NMSE between input and output (1000 km) spectrum for \ac{NFT} and
   DFT (a),(c) and percentage of energy
in the continuous \ac{NFT} spectrum (b),(d) as a function of the launch power
for noiseless (a),(b) and noisy
(c),(d) amplification.}
    % \caption{NMSE of FFT and \ac{NFT} spectra as a function of input power
    % after 1000 km of transmission.}
  \label{fig:powersweep_with_noise}
\end{figure}

\subsection{Conclusion}
\linespread{1.2}
%The effect of launch power, AWG noise and fiber loss on the \ac{NFT} continuous
% spectrum has been investigated %comparing the NMSE between ideal and distorted
% spectrum. The error has been compared against the performance %achievable using
% DFT instead of \ac{NFT}.
The impact of AWG noise on the NMSE of the \ac{NFT} and DFT shows a worsening of
performance for the \ac{NFT} when high launch powers are used. However, a higher
tolerance to nonlinearities, in terms of NMSE between input and output spectrum,
is reported for the \ac{NFT}. Lower values are shown compared to the DFT.
Finally when lossy and noisy transmission is considered, an improved spectral
matching can be achieved by the \ac{NFT}. A decrease of the NMSE of almost three
times as well as an increase in launch power by 3 dB compared to the DFT is
demonstrated.
% However, the \ac{NFT} shows a higher tolerance to loss with the NMSE between
% input and output spectrum for % noiseless transmission showing lower values for
% the \ac{NFT} compared to the DFT. Additionally, when lossy and noisy
% transmission is considered, a better % spectral matching can be achieved using
% the \ac{NFT} compared to the DFT, enabling a decrease of the NMSE of % almost
% three times and an increase in launch power by 3~dB.

