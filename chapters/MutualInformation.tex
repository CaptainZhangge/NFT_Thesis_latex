Citare paper Nikita...
b
Digital communication systems are usually analysed in respect to their capacity, which gives the maximum number of information bits that can be transmitted per channel use with an arbitrary low error probability. More noise and thus lower \ac{SNR} lead to lower capacity. The capacity $C$ of a communication system with a linear channel and \ac{AWGN}  is given by a closed form formula \cite{shannon2001mathematical}:
\begin{equation}
\label{eq:capacity}
	C = \log_2\left( 1+\text{SNR} \right).
\end{equation}

\subsubsection{Nonlinear Frequency Division Multiplexed System}
In this section we present the results on the AIR for a \ac{NFDM} system using the discrete part of the spectrum of the \ac{NFT}.

% \subsection{System description}
% At the transmitter the data bits are mapped to the \scatcoef{} $\nftb[][i]$ for i=1,2
% where the pair of eigenvalues $\{\eig[1] = 0.3j,~\eig[2] = 0.6j\}$ is used for each symbol. The NFT coefficients associated to the first eigenvalue can assume values drawn from a \ac{QPSK} constellation of radius 5 while those associated to the second eigenvalue are drawn from a \ac{QPSK} constellation of radius 0.14 and phase shifted by $\pi/4$. The time-domain signal is then generated by performing an inverse NFT using the \ac{DT}. \ac{AWGN} noise is then added to the waveform for various levels of \ac{SNR}. Given that the signal power of the \ac{NFT} signal cannot be varied freely without altering its time duration, the noise power was varied in order to vary the \ac{SNR}. Finally the \ac{NFT} of the noisy signal is performed symbol by symbol to obtain the received eigenvalues  $\{\eigest[1],~\eigest[2]\}$ and the corresponding \scatcoef{} $\{\nftbest[][1], \nftbest[][2]\}$.

\noindent Systems with non-linear channel are more difficult to evaluate as for the case of the non-linear fiber channel in optical communication systems~\cite{essiambre2010capacity}. Here, the signal propagation through the fiber is governed by the non-linear Schr\"{o}dinger equation. For this reason, an exact closed form solution for the capacity is unknown.
Closed form capacity formulas are based on approximations, e.g. the Gaussian noise model. It assumes that the non-linear interference noise is an additional Gaussian noise source. If an exact solution is required it is necessary to fall back to simulations/experiments and calculate an estimate of the capacity from the actual received data. Such a method calculates an estimate of the capacity in form of the \ac{AIR}.\\
If the system is memory-less the transmitted/received symbol sequences are given by the random variables $X$ and $Y$ and the capacity is lower bounded by:

% .EQ. Capacity


\begin{align}\label{eq:air}
	C &\geq I(X,Y) \\
	&=  \expectation_{p(X,Y)}\left[\log \dfrac{q_{Y|X}(Y|X)}{q_{Y}(Y)}\right] \\
	&= \sum_{a} \int_{b}q_{X,Y}(a,b)\,\log \dfrac{q_{Y|X}(b|a)}{q_{Y}(b)}db \\
	&= \sum_{a} p_{X}(a)\int_{b}\,q_{Y|X}(b|a)\,\log \dfrac{q_{Y|X}(b|a)}{q_{Y}(b)} db\\
	&= \sum_{a} p_{X}(a)\int_{b}\,q_{Y|X}(b|a)\,\log \dfrac{q_{Y|X}(b|a)}{\sum_{a} q_{X,Y}(a,b)} db\\
	&= \sum_{a} p_{X}(a)\int_{b}\,q_{Y|X}(b|a)\,\log \dfrac{q_{Y|X}(b|a)}{\sum_{a} p_{X}(a)q_{Y|X}(b|a)} db
\end{align}
where $p(\cdot)$ represent a known probability distribution of the system and $q(\cdot)$ represent a "guessed" probability distribution used in place of the real one which is unknown.
\noindent Since the transition probability distributions of the channel $p_{Y|X}(b|a)$ are not available, they must be estimated from the simulated data. Thus, from the data we estimate the histograms that will be used in place of the
real probability distributions of the channel:
\begin{equation}
\label{eq:hist}
\begin{split}
	q_{Y|X}(b|a) = \dfrac{q_{X,Y}(a,b)}{q_{X}(a)} = \frac{ N(a,b) }{ N(a) }
\end{split}
\end{equation}
where $N(\cdot)$ represents the number of occurrences of the given symbol or symbols pair.
Now, the histogram or a fit of the histogram can be used to estimate the lower bound of the capacity in \eqref{eq:air}. The MATLAB code to estimate the a posteriori probability distributions and to compute the \ac{AIR} is given in the code section.

\subsection{Assumptions}
In order to compute the \ac{AIR} some assumptions have been made.

\paragraph{SNR}
The \ac{AWGN} channel model assumes that the signal at the receiver is matched-filter in order to optimally remove the maximum amount of noise possible. The \ac{SNR} is computed after this operation. In \ac{NFDM} systems the matched filter is not known. For this reason an ideal rectangular filter with bandwidth $B_f$ is used to filter the out of band noise. The bandwidth $B_f$ is set to be equal to the 95\% single-sided power bandwidth of the transmitted signal, which is 4.25~GHz (computed as half the value of the baseband bandwidth returned by the \codeword{obw} command of MATLAB). In this way the \ac{SNR} can be computed as:

%%%% .FIG. SPECTRUM NFDM %%%%
\begin{equation}
SNR = \dfrac{Ps}{N_0 B_f}
\end{equation}
where $P_s$ is the noise power and $N_0$ is the \ac{AWGN} density.
In \figurename~\ref{fig:spectrum_nft} the spectrum of the signal with and without noise is shown.

\begin{figure}[!h]
  \centering
  \includegraphics[width=.6\textwidth]{./img/spectrum.png}
  \caption{The spectrum of the noise-free signal and its noisy version is shown in blue and red respectively. The light blue shaded area represents the 95\% baseband power bandwidth of the noise-free signal as computed with the \codeword{obw} command of MATLAB.}
  \label{fig:spectrum_nft}
\end{figure}

% \paragraph{Uncertainty on the eigenvalues}
% We can make a parallel between an \ac{NFDM} system and an \ac{OFDM} system, where the eigenvalues of the first correspond to the frequencies of the second and the \scatcoef{} correspond to the Fourier coefficients. There are few main differences though: in the \ac{OFDM} case the receiver knows where the frequencies of the subcarriers are, while the \ac{NFDM} receiver needs to determine the positions of the eigenvalues from the received time-domain signal. These positions can vary as a consequence of the presence of the noise. Moreover the \ac{NFDM} receiver do not locate the eigenvalues in order in general. Because of this, a decision on the eigenvalues needs to be made in order to sort them, and the corresponding \scatcoef{}, before taking a decision on the received symbols. This is also necessary to assign the symbols correctly to the two "subchannels" we have considered to compute the \ac{AIR} of the system. An example of received eigenvalues and \scatcoef{} is shown in \figurename~\ref{fig:mixed_eigen}.
%
% %%%% .FIG. MIXED EIGEN %%%%
% \begin{figure}[!htb]
%   \centering
%   \includegraphics[width=\textwidth]{./img/mixed_eigen.png}
%   \caption{Example of received eigenvalues and scattering coefficients. The highlighted line shows a case where they are not ordered properly.}
%   \label{fig:mixed_eigen}
% \end{figure}

\paragraph{NFDM as multi channel system}
Following the parallelism between an \ac{NFDM} system and an \ac{OFDM} system the \scatcoef{} associated to the two eigenvalues have been considered as two independent "subchannels", so that the total \ac{AIR} of the system is the sum of the \ac{AIR} of the two "subchannels". It should be noted that this is a strong assumption that may not hold true. Indeed, the two "subchannels" in this case share the same portion of the spectrum so that some correlation may be present among them, which may lower the total \ac{AIR}. In this sense the computed \ac{AIR} can be thought as an optimistic estimation of the real one.

\subsection{\ac{AIR} of the system}
Given the previous assumptions the \ac{AIR} can be computed as:

\begin{equation}
\label{eq:nft_capacity}
\begin{split}
    C& \ge  \sum_{i=1}^{2} I(X_i,Y_i)\\
	&= \sum_{i=1}^{2} \sum_{a} p_{X_i}(a)\int_{b}\,q_{Y_i|X_i}(b|a)\,\log \dfrac{q_{Y_i|X_i}(b|a)}{\sum_{a} p_{X_i}(a)q_{Y_i|X_i}(b|a)} db
\end{split}
\end{equation}
where $X_i = \nftb[][i]$ and $Y_i = \nftbest[][i]$ for $i=1,2$.

In \figurename~\ref{fig:air_nft} the \ac{AIR} for the \ac{NFDM} system is depicted together with the computed \ac{BER}.
%%%% .FIG. AIR NFT %%%%
\begin{figure}[!htb]
  \centering
  \includegraphics[width=\textwidth]{./img/AIR_BER.pdf}
  \caption{In the top of the figure the \ac{AIR} of the \ac{NFDM} system is shown together with the \ac{AWGN} capacity boundary. The \ac{AIR} of a standard 16-\ac{QAM} coherent system is also shown for comparison.
  In the lower part of the figure the \ac{BER} of the system is shown.}
  \label{fig:air_nft}
\end{figure}
