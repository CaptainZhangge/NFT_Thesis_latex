%%%%%%%%%%%%%%%%%%%% ABBREVIATIONS %%%%%%%%%%%%%%%%%%%%%%%%%%
% Units
\newcommand{\gbaud}{GBd}

% Terms
\newcommand{\scatcoef}{scattering coefficients}
\newcommand{\zs}{Zakharov-Shabat}
\newcommand{\mz}{Mach-Zehnder}
\newcommand{\nfdmsymbol}{\ac{NFDM}-symbol}

%%%%%%%%%%%%%%%%%%%%% COMMANDS %%%%%%%%%%%%%%%%%%%%%%%%%%%%%%
% Latex
\newcommand{\figref}[1]{\figurename~\ref{#1}}       % Figure reference as 'Figure 1'
\newcommand{\tabref}[1]{\tablename~\ref{#1}}        % Table reference as 'Table 1'
\newcommand{\chref}[1]{\chaptername~\ref{#1}}       % chapter reference as 'Table 1'
\newcommand{\codeword}[1]{\texttt{{#1}}}            % Italiic for pieces of code

% Figures
\newcommand{\figuresvspace}{\vspace{0.2cm}}
\newcommand{\figuresvspacenl}{\vspace{0.2cm}\newline}

% General math
\newcommand{\RR}{\mathbb{R}}                % Set of real numbers
\newcommand{\CC}{\mathbb{C}}                % Set of complex numbers
\newcommand{\iunit}{i}                      % Imaginary unit

% Operators
\DeclareMathOperator*{\expectation}{E}      % Expectation
\DeclareMathOperator*{\argmin}{argmin}      % argmin
\DeclareMathOperator*{\argmax}{argmax}      % argmax
\DeclareMathOperator*{\nmse}{NMSE}          % Normalized mean squared error
\DeclareMathOperator*{\papr}{PAPR}          % Peak-to-average power ratio
\DeclareMathOperator*{\osnr}{OSNR}          % Optical signal-to-noise ratio
\DeclareMathOperator*{\snr}{SNR}            % Signal-to-noise ratio
\DeclareMathOperator*{\diagmatr}{diag}      % Diagonal matrix inline


\newcommand{\matr}[1]{\mathbf{#1}}

% NLSE/Manakov
\newcommand{\ttm}{\tau}                     % Time
\newcommand{\ssp}{\ell}                     % Space
\newcommand{\fld}[1][]{E_{#1}}              % Electrical field/signal
\newcommand{\flds}[1][]{\fld[#1]            % Field/signal  with time coordinate
                (\ttm)}
\newcommand{\nttm}{t}                       % Normalized time
\newcommand{\nssp}{z}                       % Normalized space
\newcommand{\nfld}[1][]{q_{#1}}             % Normalized field/signal
\newcommand{\nflds}[1][]{\nfld[#1]          % Normalized field/signal  with time coordinate
                (\nttm)}
\newcommand{\nflddig}[1][]{\nfld[#1]          % Normalized field/signal  with discrete time coordinate
                (\nttm_n)}
\newcommand{\nfldl}[1][]{\nfld[#1]          % Normalized field/signal  with coordinates
                (\nttm, \nssp)}
\newcommand{\loss}{\alpha}                  % Loss coefficient (alpha)
\newcommand{\dispersion}{\beta_2}           % Dispersion coefficient (beta2)
\newcommand{\nonlinfact}{\gamma}            % Nonlinear coefficient (gamma)
\newcommand{\fiberl}{L}                     %  Fiber length
\newcommand{\spanl}{L_{s}}                  %  Span length
\newcommand{\nspans}{N_s}                   %  Number of spans

% NFDM system
\newcommand{\Rs}{$R_{s}$}                   % Symbol rate
\newcommand{\Ts}{$T_{s}$}                   % Symbol period
\newcommand{\Tsi}[1]{$T_{s,#1}$}            % Symbol period i-th
\newcommandx{\sTx}[2][1=,2=]{\ifthenelse{   % Transmitted waveform
  \equal{#1}{}}
    {\fld[Tx](\ttm_{#2})}
    {\fld[Tx,#1](\ttm_{#2})}
  }
\newcommandx{\sRx}[2][1=,2=]{\ifthenelse{   % Received waveform
  \equal{#1}{}}
    {\fld[Rx](\ttm_{#2})}
    {\fld[Rx,#1](\ttm_{#2})}
  }
\newcommand{\sdTx}[1][]{\sTx[#1][n]}          % Transmitted digital waveform
\newcommand{\sdRx}[1][]{\sRx[#1][n]}          % Received digital waveform
\newcommand{\nfdmsymtx}{\{\eig[1], \dots, \eig[N], \nftb[][1], \dots, \nftb[][N]\}}
\newcommand{\nfdmsymrx}{{\{\eigest[1], \dots, \eigest[N], \nftbest[][1], \dots, \nftbest[][N]\}}}
% Other optical terms
\newcommand{\phnoise}{\sigma_n}             % Phase noise variance
\newcommand{\asenoise}{n}                   % ASE noise term

% NFT
\newcommand{\nftcvL}{\mathcal{L}}           % NFT characteristic length
\newcommand{\To}{$T_{0}$}                   % NFT characteristic time

\newcommand{\eig}[1][]{\lambda_{#1}}                  % Eigenvalue [1 = subscript index]
\newcommand{\eigf}[1][]{v_{#1}}                       % Eigenfunction [1 = subscript index]
\newcommand{\vecv}{v}  % Obsolete, use eigf           % Solution ZS problem
\newcommand{\nfta}[1][]{a(\eig[#1])}                  % NFT coefficient 'a' [1 = subscript index]
\newcommand{\nftaderiv}[1][]{a'(\eig[#1])}            % NFT coefficient 'a1' [1 = subscript index]
\newcommandx{\nftb}[2][1=, 2=]{\ifthenelse{%          % NFT coefficient 'b'[1 = mode index, 2 = subscript index]
  \equal{#2}{}}{b_{#1}(\eig)}{b_{#1}(\eig[#2])}}
\newcommand{\cnft}[1][]{\ifthenelse{                  % NFT continuous spectrum [1 = subscript index]
  \equal{#1}{}}{Q_{c}(\eig)}{Q_{c,#1}(\eig)}}
\newcommand{\dnft}[1][]{\ifthenelse{                  % NFT discrete spectrum [1 = subscript index]
  \equal{#1}{}}{Q_{d}(\eig)}{Q_{d}(\eig[#1])}}
\newcommand{\cnftsp}[1][]{Q_{c}(\eig}                 % NFT continuous spectrum with space for space var
\newcommand{\dnftsp}[1][]{\ifthenelse{
  \equal{#1}{}}{Q_{d}(\eig}{Q_{d}(\eig[#1]}}   % NFT discrete spectrum with space for space var
\newcommand{\nftH}[1][]{e^{-4\iunit\eig[#1]^2\nssp}}         % NFT spectrum space evolution / Channel transfer function
\newcommandx{\nftInvH}[2][1=, 2=]{e^{4\iunit\eig[#1]^2#2}}       % Inverse NFT spectrum space evolution (z = 1) / Inverse channel transfer function
\newcommand{\nEig}{N}                                 % Number of discrete eigenvalues
\newcommand{\bitsperblock}{d}
% Estimated at the reciever
\newcommand{\eigest}[1][]{\ifthenelse{                % Eigenvalue [1 = subscript index]
  \equal{#1}{}}
    {{\widetilde{\lambda}}}
    {{\widetilde{\lambda}_#1}}
  }
\newcommandx{\nftbest}[2][1=, 2=]{\ifthenelse{        % NFT coefficient 'b'[1 = mode index, 2 = subscript index]
  \equal{#2}{}}
    {\widetilde{b_{#1}}(\eigest)}
    {\widetilde{b_{#1}}(\eigest[#2])}
  }
\newcommand{\cnftest}[1][]{\ifthenelse{                  % NFT continuous spectrum [1 = subscript index]
  \equal{#1}{}}
    {\widehat{Q}_{c}(\eig)}
    {\widehat{Q}_{c, #1}(\lambda)}
  }
% Conjugate versions
\newcommand{\nftaconj}[1][]{\bar{a}(\eig[#1])}        % NFT coefficient 'a' conjugate [1 = subscript index]
\newcommandx{\nftbconj}[2][1=, 2=]{\ifthenelse{       % NFT coefficient 'b' conjugate [1 = mode index, 2 = subscript index]
  \equal{#2}{}}
    {\bar{b}_{#1}(\eig)}
    {\bar{b}_{#1}(\eig[#2])}
  }
% Jost solutions
\newcommand{\jost}[1][]{\phi_{#1}(t,\eig)}            % Jost solution generic
\newcommandx{\jostp}[2][1=, 2=]{\ifthenelse{                 % Jost solution p [1 = component index, 2 = eigenvalue index]
  \equal{#2}{}}
    {\phi^{P}_{#1}(t,\eig)}
    {\phi^{P}_{#1}(t,\eig[#2])}
  }
\newcommandx{\jostn}[2][1=, 2=]{\ifthenelse{                 % Jost solution n [1 = component index, 2 = eigenvalue index]
  \equal{#2}{}}
    {\phi^{N}_{#1}(t,\eig)}
    {\phi^{N}_{#1}(t,\eig[#2])}
  }
\newcommand{\jostpconj}[1][]{\bar{\phi}^{P}_{#1}      % Jost solution p conjugate
            (t,\eig)}
\newcommand{\jostnconj}[1][]{\bar{\phi}^{N}_{#1}      % Jost solution n conjugate
            (t,\eig)}
\newcommand{\jostpshrt}[1][]{\phi^{P}_{#1}}           % Jost solution p (short notation)
\newcommand{\jostnshrt}[1][]{\phi^{N}_{#1}}           % Jost solution n (short notation)
\newcommand{\jostpconjshrt}[1][]{\bar{\phi}^{P}_{#1}} % Jost solution p conjugate (short notation)
\newcommand{\jostnconjshrt}[1][]{\bar{\phi}^{N}_{#1}} % Jost solution n conjugate (short notation)

% Darboux
\newcommand{\nfldsmod}[1][]{\hat{q}_{#1}             % Normalized field/signal  with time coordinate
                (\nttm)}

\newcommand{\auxsol}{\bar{v}}           % Darboux transform auxiliary solution
\newcommand{\auxsolmat}{\Theta} % Darboux transform auxiliary solution matrix

% NFT math relateed quantities
\newcommand{\wron}[2]{W(#1, #2)} % Wronskian

% NFT math relateed quantities - Single pol
% \newcommand{\vv}{\begin{pmatrix} \psi_1\\ \psi_2\end{pmatrix}} % Obsolete
% \newcommand{\vvt}{\renewcommand*{\arraystretch}{1.2}\begin{pmatrix} \frac{\partial\psi_1}{\partial\tau}\\ % Obsolete
% \frac{\partial\psi_2}{\partial\tau}\end{pmatrix}}
\newcommand{\matL}{\boldsymbol{L}}
\newcommand{\matP}{\boldsymbol{P}}
\newcommand{\matM}{\boldsymbol{M}}
\newcommand{\matPfull}{\begin{pmatrix} -j\eig & q(t) \\ -q(t)^* & j\eig \end{pmatrix}}

% NFT math relateed quantities - Dual pol
\newcommand{\mzssol}[1][]{v_{#1}}                        % Vector solution ofthe ZSP [1 = component number] (Was \dv)
% \newcommand{\dvv}{\begin{pmatrix} % Obsolete
%         \psi_1\\ \psi_2\\ \psi_3\end{pmatrix}} % Obsolete
% \newcommand{\dvvt}{\renewcommand*{\arraystretch}
%         {1.2}\begin{pmatrix} \frac{\partial\psi_1}
%         {\partial\tau}\\
% \frac{\partial\psi_2}{\partial\tau}\end{pmatrix}}

% Forward backward method
\newcommand{\psisol}[1][]{\psi_{#1}}               % Solution of the ZSP after the change of variable [1 = component number]
\newcommand{\fbw}[1][]{w_{#1}}                     % Solution of the ZSP after the change of variable [1 = component number]
\newcommand{\fbu}[1][]{u_{#1}}                     % Solution of the ZSP after the change of variable [1 = component number]

\newcommand{\tjostp}[1][]{\psi^{P}_{#1}(t,\eig)}   % Transformed Jost solution p
\newcommand{\tjostn}[1][]{\psi^{N}_{#1}(t,\eig)}   % Transformed Jost solution n
\newcommand{\tjostpshrt}[1][]{\psi^{P}_{#1}}       % Transformed Jost solution p (short notation)
\newcommand{\tjostnshrt}[1][]{\psi^{N}_{#1}}       % Transformed Jost solution n (short notation)

% From NFT tutorial paper (should be removed and not used)
\newcommand{\wrong}[1]{{\color{red}#1}}
\newcommand{\ms}[1]{\mathds{#1}}
\newcommand{\Ex}{\ms{E}}
\newcommand{\Ns}{n_{\tnr{s}}}
\newcommand{\tnr}[1]{{\textnormal{#1}}}
\newcommand{\ld}{\ldots}
\newcommand{\set}[1]{\{#1\}}
\newcommand{\mcXkb}{\mathcal{X}_{k}^{b}}
\newcommand{\mcXko}{\mathcal{X}_{k}^{1}}
\newcommand{\mcXkz}{\mathcal{X}_{k}^{0}}
\newcommand{\mcX}{\mathcal{X}}

% Command to reference a figure inset
\newcommand{\insetref}[1]{(\textbf{#1})}

%%%%%%%%%%%%%%%%%%%%% COMMANDS RENEWED %%%%%%%%%%%%%%%%%%%%%%

% Compact matix
\newcommand{\compactmat}{\renewcommand                % Reduce the vertical space inside matrices
        {\arraystretch}{1}}

% Equation number with S1 style (for supplementary information)
% \renewcommand{\theequation}{S.\arabic{equation}}

% System of equations with a single equation number
\newenvironment{nalign}{                              % Align environment with a single equation number for multiple equations
    \begin{equation}
    \begin{aligned}
}{
    \end{aligned}
    \end{equation}
    \ignorespacesafterend
}


% Make my name bold (needs to be annotated in the bibliography)
\renewcommand*{\mkbibnamegiven}[1]{%
\ifitemannotation{simonegaiarin}
{\textbf{#1}}
{#1}}

\renewcommand*{\mkbibnamefamily}[1]{%
\ifitemannotation{simonegaiarin}
{\textbf{#1}}
{#1}}

\newcommand{\symb}[1]{\item[${#1}$]}
\newcommand{\symbt}[1]{\item[{#1}]}
